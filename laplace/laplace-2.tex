\documentclass{bkcnotes}
\usepackage{bkc}

\setcourse{Partial Differential Equations}
\setnotetitle{Laplace Equation II\ : \\ The Fundamental Solution}
\setnoteimage{img/partial}

\newcommand{\todo}[1]{{\bf TODO: } #1\\}

\begin{document}
\maketitle

Ok. Let's get down to business

\section{Solving the Laplace Equation}
We want to solve Laplace's equation
\begin{equation}
  \label{eq:laplace}
  \lap u = 0
\end{equation}
and discover as much as we can about the properties of the solutions
before coming across an explicit representation formula. The solutions
of \eqref{eq:laplace} have a special name

\begin{ndefn}(Harmonic Function)
  
  A $C^2$ function satisfying \eqref{eq:laplace} is called
  \textbf{harmonic}.
\end{ndefn}

With the transport equation our strategy for the solution consisted of
two steps:
\begin{enumerate}
\item Identify a symmetry property of the solution.
\item Use the symmetry to reduce the PDE to an ODE.
\item Solve the ODE.
\end{enumerate}
For Laplace, we are going to run with this again. Before we had
\emph{translational symmetry} and the ODE was a simple one $v' = f$
(in the nonhomogeneous case). this time around things will be a bit
messier, but the same principles are at work.

\subsection{Symmetry Properties}
Motivated by the discussion in the derivation last time, we would
expect the solutions to have \emph{rotational symmetry} because points
equidistant from the same mass should experience the same
gravitational force from that mass. Let's formalize this
intution. Recall that rotation matrices belong to a special matrix
group

\begin{ndefn}(Special Orthogonal Group)
  
  The \textbf{special orthogonal group}\footnote{Verify that this set
    in fact merits the word "group"} $O(n)$ consists of all square $n
  \times n$ matrices, $A$ satisfying:
  \begin{itemize}
  \item $A^tA = I$
  \item $\det A = \pm 1$
  \end{itemize}
\end{ndefn}
We want to verify the following
\begin{nprop}(Rotational Symmetry of Laplace Equation)

  Suppose $u$ satisfies \eqref{eq:laplace}. Then for all $Q \in O(n)$
  the function
  \begin{equation}
    \label{eq:w}
    w(x) = u(Qx)
  \end{equation}
  is also satisfies \eqref{eq:laplace}.
\end{nprop}
\begin{proof}
  This boils down to a computation. For a start,
  \begin{equation}
    \label{eq:w-grad-comp}
    \begin{aligned}
      \grad w(x) \cdot \eta &= \frac{d}{d\alpha}
      u(Q(x+\alpha\eta))\mid_{\alpha=0}
      && (\forall\eta \in \R^n) \\
      &= \grad u(Qx)\cdot Q\eta \\
      &= Q^t\grad u(Qx)\cdot \eta
    \end{aligned}
  \end{equation}

  And so
  \begin{equation}
    \label{eq:grad-w}
    \grad w(x) = Q^t\grad u(Qx)
  \end{equation}
  And for the second gradient we get
  \begin{equation}
    \label{eq:w-snd-grad-comp}
    \begin{aligned}
      \grad^2w(x)\eta &= \frac{d}{d\alpha}(Q^t\grad
      u(Q(x+\alpha\eta)))\mid_{\alpha=0} \\
      &= Q^t\grad^2u(Qx)Q\eta
    \end{aligned}
  \end{equation}
  Consequently,
  \begin{equation}
    \label{eq:w-snd-grad}
    \grad^2w(x) = Q^t\grad^2(Qx)Q
  \end{equation}
  Which finally gives us
  \begin{equation}
    \label{eq:lap-w}
    \begin{aligned}
      \lap w(x) &= \trace (Q^t\grad^2u(Qx)Q) \\
      &= \trace (\underbrace{Q^tQ}_{I}\grad^2u(Qx))
      && (\text{because } \trace (AB) = \trace(BA)) \\
      &= \lap(Qx) \\
      &= 0
    \end{aligned}
  \end{equation}
  As desired.
\end{proof}

\subsection{Reduction to an ODE}
Step one complete. Now that we have verified our symmetry property, we
need to use this to find solutions of the form $u(x) = v(|x|)$. Where
it's convenient we will let $r = |x|$. This will lead us to the ODE we
need, completing the second step in our plan. It's another computation
grind
\begin{equation}
  \label{eq:grad-u-comp}
  \begin{aligned}
    \grad u(x) \cdot \eta &= \frac{d}{d\alpha} v([(x+\alpha\eta)\cdot
    (x+\alpha\eta)]^{1/2}])\mid_{\alpha = 0} \\
    &= v'(|x|)\left(\frac{1}{2}\frac{2x\cdot\eta}{|x|}\right)
  \end{aligned}
\end{equation}
Which implies
\begin{equation}
  \label{eq:grad-u}
  \grad u(x) = \frac{v'(|x|)}{|x|}x
\end{equation}
Let
\begin{equation}
  \label{eq:g-def}
  g(|x|) = \frac{v'(|x|)}{|x|}
\end{equation}

\subsection{Solving the ODE}

\section{Properties of the Fundamental Solution}
\end{document}