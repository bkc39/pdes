\documentclass{bkcnotes}
\usepackage{bkc}

\setcourse{Partial Differential Equations}
\setnotetitle{Laplace Equation II\ : \\ The Fundamental Solution}
\setnoteimage{img/partial}

\newcommand{\todo}[1]{{\bf TODO: } #1\\}

\begin{document}
\maketitle

OK. Let's get down to business

\section{Solving the Laplace Equation}
We want to solve Laplace's equation
\begin{equation}
  \label{eq:laplace}
  \lap u = 0
\end{equation}
and discover as much as we can about the properties of the solutions
before coming across an explicit representation formula. The solutions
of \eqref{eq:laplace} have a special name

\begin{ndefn}(Harmonic Function)
  
  A $C^2$ function satisfying \eqref{eq:laplace} is called
  \textbf{harmonic}.
\end{ndefn}

With the transport equation our strategy for the solution consisted of
two steps:
\begin{enumerate}
\item Identify a symmetry property of the solution.
\item Use the symmetry to reduce the PDE to an ODE.
\item Solve the ODE.
\end{enumerate}
For Laplace, we are going to run with this again. Before we had
\emph{translational symmetry} and the ODE was a simple one $v' = f$
(in the nonhomogeneous case). this time around things will be a bit
messier, but the same principles are at work.

\subsection{Symmetry Properties}
Motivated by the discussion in the derivation last time, we would
expect the solutions to have \emph{rotational symmetry} because points
equidistant from the same mass should experience the same
gravitational force from that mass. Let's formalize this
intuition. Recall that rotation matrices belong to a special matrix
group

\begin{ndefn}(Special Orthogonal Group)
  
  The \textbf{special orthogonal group}\footnote{Verify that this set
    in fact merits the word "group"} $O(n)$ consists of all square $n
  \times n$ matrices, $A$ satisfying:
  \begin{itemize}
  \item $A^tA = I$
  \item $\det A = \pm 1$
  \end{itemize}
\end{ndefn}
We want to verify the following
\begin{nprop}(Rotational Symmetry of Laplace Equation)

  Suppose $u$ satisfies \eqref{eq:laplace}. Then for all $Q \in O(n)$
  the function
  \begin{equation}
    \label{eq:w}
    w(x) = u(Qx)
  \end{equation}
  is also satisfies \eqref{eq:laplace}.
\end{nprop}
\begin{proof}
  This boils down to a computation. For a start,
  \begin{equation}
    \label{eq:w-grad-comp}
    \begin{aligned}
      \grad w(x) \cdot \eta &= \frac{d}{d\alpha}
      u(Q(x+\alpha\eta))\mid_{\alpha=0}
      && (\forall\eta \in \R^n) \\
      &= \grad u(Qx)\cdot Q\eta \\
      &= Q^t\grad u(Qx)\cdot \eta
    \end{aligned}
  \end{equation}

  And so
  \begin{equation}
    \label{eq:grad-w}
    \grad w(x) = Q^t\grad u(Qx)
  \end{equation}
  And for the second gradient we get
  \begin{equation}
    \label{eq:w-snd-grad-comp}
    \begin{aligned}
      \grad^2w(x)\eta &= \frac{d}{d\alpha}(Q^t\grad
      u(Q(x+\alpha\eta)))\mid_{\alpha=0} \\
      &= Q^t\grad^2u(Qx)Q\eta
    \end{aligned}
  \end{equation}
  Consequently,
  \begin{equation}
    \label{eq:w-snd-grad}
    \grad^2w(x) = Q^t\grad^2(Qx)Q
  \end{equation}
  Which finally gives us
  \begin{equation}
    \label{eq:lap-w}
    \begin{aligned}
      \lap w(x) &= \trace (Q^t\grad^2u(Qx)Q) \\
      &= \trace (\underbrace{Q^tQ}_{I}\grad^2u(Qx))
      && (\text{because } \trace (AB) = \trace(BA)) \\
      &= \lap(Qx) \\
      &= 0
    \end{aligned}
  \end{equation}
  As desired.
\end{proof}

\subsection{Reduction to an ODE}
Step one complete. Now that we have verified our symmetry property, we
need to use this to find solutions of the form $u(x) = v(|x|)$. Where
it's convenient we will let $r = |x|$. This will lead us to the ODE we
need, completing the second step in our plan. It's another computation
grind
\begin{equation}
  \label{eq:grad-u-comp}
  \begin{aligned}
    \grad u(x) \cdot \eta &= \frac{d}{d\alpha} v([(x+\alpha\eta)\cdot
    (x+\alpha\eta)]^{1/2}])\mid_{\alpha = 0} \\
    &= v'(|x|)\left(\frac{1}{2}\frac{2x\cdot\eta}{|x|}\right)
  \end{aligned}
\end{equation}
Which implies
\begin{equation}
  \label{eq:grad-u}
  \grad u(x) = \frac{v'(|x|)}{|x|}x
\end{equation}
Let
\begin{equation}
  \label{eq:g-def}
  g(|x|) = \frac{v'(|x|)}{|x|}
\end{equation}
Now we use our trace trickery to finish computing the laplacian in
terms of $g$. We get
\begin{equation}
  \label{eq:snd-grad-comp}
  \begin{aligned}
    \grad^2u(x) \eta &= \frac{d}{d\alpha}\left(g((x +
      \alpha\eta)\cdot(x +
      \alpha\eta)^{1/2})(x+\alpha\eta)\right)\mid_{\alpha = 0} \\
    &= g'(x)\left(\frac{1}{2}\frac{2x\cdot\eta}{|x|}\right) x + g(|x|)\eta \\
    &= g(x)I + \frac{g'(|x|)}{|x|}(x \otimes x)
  \end{aligned}
\end{equation}
Doing some calculus, we also see
\begin{equation}
  \label{eq:g-prime-def}
  g'(r) = \frac{v''(r)}{r} - \frac{v'(r)}{r^2}
\end{equation}
Throwing \eqref{eq:g-prime-def} into \eqref{eq:snd-grad-comp} we get
\begin{equation}
  \label{eq:snd-grad}
  \grad^2u(x) =
  \frac{v'(|x|)}{|x|}I + \left(v''(|x|) - 
    \frac{v'(|x|}{r}\right)\left(\frac{x}{|x|} \otimes \frac{x}{|x|}\right)
\end{equation}
We are close now. Just hit \eqref{eq:snd-grad} with the trace operator
and we get
\begin{equation}
  \label{eq:laplace-ode}
  \lap u = \trace\grad^2 u
  = \frac{nv'(r)}{r} - \left(v''(r) - \frac{v'(r)}{r}\right)
  = v'' + \frac{(n-1)}{r}v'
\end{equation}
So although it required a bit more coaxing than our friend the
transport equation, we can still reduce the PDE to an ODE.
\subsection{Solving the ODE}
Two steps down. Fortunately, this part is straightforward. We have to
solve the ODE \eqref{eq:laplace-ode}, namely
\begin{equation}
  \label{eq:ode-to-solve}
  v'' + \frac{(n-1)}{r}v' = 0
\end{equation}
Before solving this, note that the qualitative behavior of the
solutions will depend on $n$. Also, we should expect that $r$ in the
denominator to cause some problems for the solutions as $r \to
0$. With that in mind, let's dig in.

We begin by making the problem even easier, change variables with
$\rho = v'$ and \eqref{eq:ode-to-solve} becomes
\begin{equation}
  \label{eq:ode-first-order}
  \rho' + \frac{n-1}{r}\rho = r\rho' + (n-1)\rho = 0
\end{equation}
This is easy enough to solve at this point. I'll save you the time, we
get solutions looking like
\begin{equation}
  \label{eq:ode-soln}
  v(r) =
  \begin{cases}
    \alpha_1\ln{r} + \alpha_2 & n=2 \\
    \frac{\alpha_1}{r^{n-2}} + \alpha_2 & n \geq 3
  \end{cases}
\end{equation}

\section{The Fundamental Solution}
Hopefully the preceeding considerations motivate the following
\begin{ndefn}(Fundamental Solution)

  Let $\omega_n$ denote the volume of the unit ball in $\R^n$. The
  function
  \begin{equation}
    \label{eq:fundamental-solution}
    \Phi(x) =
    \begin{cases}
      \frac{1}{2\pi}\ln{|x|} & n=2 \\
      \frac{1}{n(n-2)\omega_n |x|^{n-2}} & n \geq 3
    \end{cases}
  \end{equation}
  is called the \textbf{fundamental solution} of the Laplace equation.
\end{ndefn}
Whoa! You may be asking "Why this choice of constants?". It turns out
that is what they have to be in order for this to really qualify for
the term "fundamental solution". More on that later. Another quick
remark on the definition, does this thing really merit the term
\emph{the} fundamental solution? Turns out the answer is "yes". More on
that later as well. This notion of fundamental solution is going to be
a recurring theme in our study of these classical PDEs.

\subsection{Preliminary Observations}
This is mostly a recap of how we got to this point. We followed the
same outline as in the transport equation: look for symmetry, use it
to reduce the PDE to an ODE and then solve that. Unlike the transport
equation, however, \eqref{eq:fundamental-solution} is a little bit
more cryptic. And how we can use it to generate other solutions is
more unclear still. For now, let's just note that $\Phi$ is
well-defined in $\R^n$ except for the origin. Moreover, everywhere it
is defined, it satisfies $\lap \Phi = 0$. The key thing about that
last observation is that
\begin{equation}
  \label{eq:fundamental-solution-convolve}
  \lap_x\Phi(x-y) = 0, x\neq y
\end{equation}
We are going to get a lot of mileage out of
\eqref{eq:fundamental-solution-convolve}. The looks of that should
start turning the convolution engine and we can get going on
constructing new solutions via a convolution. This is in fact the next
plan of action.
\end{document}