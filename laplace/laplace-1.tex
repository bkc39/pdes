\documentclass{bkcnotes}
\usepackage{bkc}

\setcourse{Partial Differential Equations}
\setnotetitle{Laplace Equation I\ : \\ Derivation}
\setnoteimage{img/partial}

\newcommand{\todo}[1]{{\bf TODO: } #1\\}

\begin{document}
\maketitle

Now we begin our study of the laplace equation. This is arguably the
most important linear PDE and so strap in, we are gonna be grinding on
this for a while. The plan is to go through some of the historical
motivation before trying to find some solutions. With the laplace
equation we will be able to deduce all sorts of properties of the
solutions without having a concrete represenation for the
solution. Finally, we will come up with some representation formulae
for solutions to round out our efforts. This is a long, uphill fight
but it will definitely be worth it.

\section{Historical Motivation for Laplace \\ and Poisson's equations}

Our plan is to derive the Laplace and it's nonhomogeneous counterpart,
the Poisson equation, from the Newtonian theory of
gravitation. Consider a test particle $m$ moving about a fixed
gravitating center $M$ in $\R^3$.

\todo{insert picture}

The particle of mass $m$ experiences an attractive force proportional
to the inverse square of the displacement
\begin{equation}
  \label{eq:inverse-square}
  f = \frac{-\rho}{|x-y|^2}\underbrace{\frac{x-y}{|x-y|}}_{\text{unit vector}}
\end{equation}
with $\rho$ constant.

\begin{ndefn}(Gravitational Potential)

  The \textbf{gravitational
    potential} with respect to $x$ of the mass $m$ at the point $y$ is
  \begin{equation}
    \label{eq:potential}
    u = \frac{\rho}{|x-y|}
  \end{equation}
\end{ndefn}

\subsection{Some Vector Calculus}
We want to compute $\nabla_x(u)$. Recall that if $u \in C^1(\Omega), v
\in C^1(\Omega, \R^3)$ for some domain $\Omega \subset \R^3$ this means
we have continuous vector fields $\nabla u(x)$ and $\nabla v(x)$ for
fixed $x$ such that
\begin{equation}
  \label{eq:grad-def}
  \begin{aligned}
    u(x+\eta) - u(x) &= \nabla u(x)\cdot \eta + o(|\eta |) \\
    v(x+\eta) - v(x) &= \nabla v(x) \eta + o(|\eta |)
  \end{aligned}
\end{equation}
for all $\eta \in \R^3$. In particular, all the directional
derivatives exist. Hence,
\begin{equation}
  \label{eq:directional-ders}
  \begin{aligned}
    \nabla u(x)\cdot \eta &= \frac{d}{d\alpha} u(x+\alpha \eta) \mid_{\alpha=0} \\
    \nabla v(x)\eta &= \frac{d}{d\alpha} v(x+\alpha \eta) \mid_{\alpha=0}
  \end{aligned}
\end{equation}
To make the computation nicer, set $y = 0$ in
\eqref{eq:potential}. So
\begin{equation}
  \label{eq:potential-gradient}
  \begin{aligned}
    \nabla\left(\frac{1}{|x+\alpha\eta |}\right)\mid_{\alpha=0} &=
    \frac{d}{d\alpha}\left[(x+\alpha\eta)\cdot (x+\alpha\eta)\right]^{-1/2} \mid_{\alpha =0} \\
    &= \frac{-1}{2}\left(\frac{2x\cdot \eta)}{(x\cdot x)^{3/2}}\right) \\
    &= \frac{-x}{|x|^3}\cdot \eta    
  \end{aligned}
\end{equation}
Plugging this back in we arrive at
\begin{equation}
  \label{eq:grad-u}
  \nabla u = \frac{-\rho}{|x-y|^2}\frac{x-y}{|x-y|} = f
\end{equation}

\subsection{Varying the Denisty}
Now suppose that the fixed body $M$ is a domain $\Omega \subset \R^3$
with distributed mass density $\rho(y) \geq 0$ in $\Omega$.

\todo{insert new figure}

Then in the case the potential is
\begin{equation}
  \label{eq:varied-potential}
  u(x) = \int_\Omega \frac{\rho(y)}{|x-y|}dy
\end{equation}
For $x \not\in \overline{\Omega}$ we get
\begin{equation}
  \label{eq:varied-grad}
  \begin{aligned}
    \nabla u &= \int_{\Omega} \rho(y) \nabla_x\left(\frac{1}{|x-y|}\right)dy \\
    &= \underbrace{-\int_{\Omega} \frac{-\rho}{|x-y|^2}\frac{x-y}{|x-y|^2}dy}_{
      \text{total gravitational force experienced by } m \text{ at } x
    }
  \end{aligned}
\end{equation}
Recall that
\begin{equation}
  \label{eq:trace-laplacian}
  \lap u = \nabla \cdot \nabla u = \trace \grad^2u
\end{equation}
Let's use the same trick as before to compute the second gradient
\begin{equation}
  \label{eq:varied-grad-comp}
  \begin{aligned}
    \grad^2 u(x)\eta &= \frac{d}{d\alpha}(\grad u(x+\alpha\eta))\mid_{\alpha = 0} \\
    &= -\frac{d}{d\alpha}\left(\frac{x+\alpha\eta}
      {[(x+\alpha\eta) \cdot (x+\alpha\eta)]^{3/2}}\right) \mid_{\alpha=0} \\
    &= -\left[ \frac{1}{|x|^3}\eta - \frac{3}{2}x\left(
        \frac{2x\cdot\eta}{|x|^5}\right)\right]
  \end{aligned}
\end{equation}
Recall that $(a \otimes b)c = a b\cdot c$ and that in
\eqref{eq:varied-grad-comp} we see the term $x x\cdot\eta$
appear. Doing the substitution gives us
\begin{equation}
  \label{eq:finish-grad-comp}
  \grad^2 u = -\frac{1}{|x|^3}\left[
    \eta - 3\left(\frac{x}{|x|} \otimes \frac{x}{|x|}\right)\eta
  \right]
\end{equation}
Which implies
\begin{equation}
  \label{eq:snd-grad}
  \grad^2 u(x) = -\frac{1}{|x|^3}\left[
    I - 3\left(\frac{x}{|x|} \otimes \frac{x}{|x|}\right)
  \right]
\end{equation}
If we plug this into \eqref{eq:varied-grad} we get
\begin{equation}
  \label{eq:snd-grad-int}
  \begin{aligned}
    \grad^2 u(x) = -\int_{\Omega} \frac{\rho(y)}{|x-y|^3}\left[
      I - 3\left(\frac{x-y}{|x-y|} \otimes \frac{x-y}{|x-y|}\right)
    \right]dy && x\not\in\partial\Omega
  \end{aligned}
\end{equation}
And by \eqref{eq:trace-laplacian} we get
\begin{equation}
  \label{eq:laplace-u}
  \begin{aligned}
    \lap u(x) &= -\int_{\Omega}\frac{\rho(y)}{|x-y|^3}\trace \left[ I
      - 3\left(\frac{x-y}{|x-y|} \otimes
        \frac{x-y}{|x-y|}\right)\right]dy
    && (x \not\in\overline{\Omega}) \\
    &= -\int_{\Omega}\frac{\rho(y)}{|x-y|^3}\left[\trace(I) - \trace
      3\left(\frac{x-y}{|x-y|} \otimes
        \frac{x-y}{|x-y|}\right)\right]dy \\
    &= -\int_{\Omega}\frac{\rho(y)}{|x-y|^3}\left[3 - 3
      \left(\frac{x-y}{|x-y|} \cdot
        \frac{x-y}{|x-y|}\right)\right]dy \\
    &= 0 && \text{because }\left(\frac{x-y}{|x-y|} \cdot
      \frac{x-y}{|x-y|} = 1\right)
  \end{aligned}
\end{equation}
So we finally arrive at
\begin{ndefn}(Laplace Equation)

  The \textbf{Laplace equation} is the second-order linear PDE
  \begin{equation}
    \label{eq:laplace}
    \lap u = 0
  \end{equation}
\end{ndefn}

\subsection{Laplace's Idea}

Laplace's motivation for this was that \eqref{eq:varied-potential} is
difficult to evaluate compute in general. However, solving his
equation \eqref{eq:laplace} was more direct and that the strategy for
understanding these gravitational phenomena should be to solve
\eqref{eq:laplace} in lieu of evaluating \eqref{eq:varied-potential}.

Another important feature of the derivation was that we required
$x\not\in\partial\Omega$ in \eqref{eq:snd-grad-int} so that we could
freely pass the derivatives inside of the integral. Later, Poisson and
Gauss showed (with restrictions on $\rho$) that
\begin{equation}
  \label{eq:poisson}
  \begin{aligned}
    -\lap u = \rho && x \in \Omega
  \end{aligned}
\end{equation}
As we will see shortly, this is much harder to prove due to not being
able to pass the $\lap$ inside the integral!
\end{document}
