\documentclass{bkcnotes}
\usepackage{bkc}

\setcourse{Partial Differential Equations}
\setnotetitle{Laplace Equation III\ : \\ Poisson's Equation}
\setnoteimage{img/partial}

\newcommand{\todo}[1]{{\bf TODO: } #1\\}

\begin{document}
\maketitle

So the title of the last notes was somewhat misleading. At this point
we have some strange function $\Phi$, not even defined in the whole
space, which we called the fundamental solution. We observed last time
that although it isn't perfect, it has some nice properties like being
harmonic where it is defined. And crucially, it is well behaved with
respect to convolution. The plan this time around is to exploit these
two properties to construct more solutions and solve the inhomogeneous
version of the Laplace equation, Poisson's equation.

\section{Some Estimates for $Phi$'s Derivatives}
$\Phi$ and its derivatives are going to be a focal point of our
discussion. That pesky singularity that $\Phi$ has means that we are
going to need some solid estimates that we can use that will help us
wrestle that singularity back into the realm of well defined
functions. Let's begin.
\end{document}
