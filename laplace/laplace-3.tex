\documentclass{bkcnotes}
\usepackage{bkc}

\setcourse{Partial Differential Equations}
\setnotetitle{Laplace Equation III\ : \\ Poisson's Equation}
\setnoteimage{img/partial}

\newcommand{\todo}[1]{{\bf TODO: } #1\\}

\begin{document}
\maketitle

So the title of the last notes was somewhat misleading. At this point
we have some strange function $\Phi$, not even defined in the whole
space, which we called the fundamental solution. We observed last time
that although it isn't perfect, it has some nice properties like being
harmonic where it is defined. And crucially, it is well behaved with
respect to convolution. The plan this time around is to exploit these
two properties to construct more solutions and solve the inhomogeneous
version of the Laplace equation, Poisson's equation.

\section{A First Attempt}
Despite the fact that we know the path we should travel to get more
solutions, let's take the scenic route and explore a reasonable but
ultimately doomed attempt. Hopefully we will find this instructive.

\subsection{Curbing the Singularity}
As we noted last time, we need to wrestle with the singularity that
$f$ has at 0. The first plan is to hit it with a nice function. Just
to establish notation, define
\begin{ndefn}(Compact Support)

  The \textbf{support} of a function $f$ is the closure of the set
  \begin{equation}
    \label{eq:support}
    \supp{f} = \{x \suchthat f(x) \neq 0\}
  \end{equation}
  A function is said to be of \textbf{compact support} if $\supp{f}$
  is bounded\footnote{It is closed by definition.}. We also will use
  the notation $C^k_c$ for the space of $k$ times continuously
  differentiable functions with compact support.
\end{ndefn}
Suppose that $f \in C_c(\R^n)$ and consider the (improper) integral
\begin{equation}
  \label{eq:i-defn}
  I = \int_{\R^n} f(x)\Phi(x)dx
\end{equation}
Since $f$ has compact support there is some ball $B_R$ centered at the
origin such that \eqref{eq:i-defn} can be estimated for $n \geq 3$ as
\begin{equation}
  \label{eq:i-estimate}
  |I| \leq \left| \int_{B_R}f(x)\Phi(x)dx\right|
  \leq (\sup_{B_R} f)\int_{B_R}\frac{1}{|x|^{n-2}}dx
\end{equation}
Because we are integrating over a ball, let's switch to polar
coordinates. Let $ds$ be the surface measure on $S^{n-1}$. 
\end{document}
