\documentclass{bkcnotes}
\usepackage{bkc}

\setcourse{Partial Differential Equations}
\setnotetitle{Transport Equation II}
\setnoteimage{img/partial}

\newcommand{\todo}[1]{{\bf TODO: } #1\\}

\begin{document}
\maketitle

Last time we found the solution to the homogeneous transport IVP, This
time around, we'll extend the intuition to the nonhomogeneous problem
as well as discuss several additional points of view on the solutions
we get that will be useful in future discussions on PDEs.

\section{Nonhomogeneous Transport Equation}

\begin{ndefn}(Nonhomogeneous Transport IVP)

  The \textbf{initial value problem for the transport equation} is
  \begin{equation}
    \label{eq:ntransport-ivp}
    \begin{cases}
      (b,1) \cdot \nabla u(x,t) = f & f \in C^1, b,x \in \R^n, t \in \R^+ \\
      u = g & t = 0, g \in C^1(\R^n)
    \end{cases}
  \end{equation}
\end{ndefn}
The intuition behind this IVP is nearly identical to
the homogeneous case, except now that instead of being constant
along $(b,1)$ we have an additional $f$ that we need to keep track
of. In essence we can still read off what the answer should be: The
initial value on the $\R^n$ plane as well as the total contribution of
$f$ along the line $(x,0)$ to $(x+sb,t+s)$. This leads us to

\begin{nthm}(Solution of Nonhomogeneous Transport Equation)

  The function
  \begin{equation}
    \label{eq:ntransport-sol}
    u(x,t) = g(x-tb) + \int_0^t f(x+(s-t)b,s)ds
  \end{equation}
  is a solution to the IVP \eqref{eq:ntransport-ivp}.
\end{nthm}
\begin{proof}
  This goes very similarly to the homogeneous case. The only
  difficulty is dealing with that $f$ term. Fortunately for us, it's
  still just calculus with a change of variables. Set
  \begin{equation}
    \label{eq:change-of-var}
    v(\tau) = u(x-\tau b,t-\tau)
  \end{equation}
  and compute
  \begin{equation}
    \label{eq:v-der}
    \begin{aligned}
      \dot{v}(\tau ) &= D_xu(x-\tau b,t-\tau)\cdot(-b) - u_t(x-\tau b,t-\tau) \\
      &= -(D_xu(x-\tau b,t-\tau)\cdot b + u_t(x-\tau b,t-\tau)) \\
      &= -f(x-\tau b,t-\tau) && \text{(by substitution in }
      \eqref{eq:ntransport-ivp})
    \end{aligned}
  \end{equation}
  But
  \begin{equation}
    \label{eq:v-int}
    \begin{aligned}
      \int_0^t \dot{v}(\tau) d\tau &= v(t) - v(0) \\
      &= \underbrace{u(x - tb,0)}_{g(x-tb)} - u(x,t)
    \end{aligned}
  \end{equation}
  So we have
  \begin{equation}
    g(x-tb) - u(x,t) = -\int_0^t f(x-\tau b, t - \tau)d\tau 
  \end{equation}
  Then apply the change of variables $-\tau = s = t$ to get
  \begin{equation}
    \begin{aligned}
      \int_0^t f(x-\tau b, t-\tau)d\tau &= -\int_t^0 f(x + (s-t)b,s)ds \\
      &= \int_0^t f(x+(s-t)b,s)ds
    \end{aligned}
  \end{equation}
  Giving us the final solution
  \begin{equation}
    \label{eq:nhom-sol}
    u(x,t) = g(x-tb) + \int_0^t f(x+(s-t)b,s)ds
  \end{equation}
\end{proof}

\subsection{Interpretation of the Solution}
Here's an interpretation of the solution. The key idea was in dinding
the solution was the fact that it had to be \emph{translation
  invariant}. This can be captured in the following operation:

\begin{ndefn}(Solution Operator to Transport Equation)

  For any $h(x,t)$ let the operator $T_b$ be given by
  \begin{equation}
    \label{eq:soln}
    [T_b(\sigma)h](x,t) = h(x-\sigma b,t)
  \end{equation}
  Then $T_b(t)$ is the \textbf{solution operator} for the homogeneous
  transport equation.
\end{ndefn}
Viewed in this light, \eqref{eq:nhom-sol} reads
\begin{equation}
  \label{eq:nhom-sol-op}
  u(x,t) = [T_b(t)g](x) + \int_0^t [T_b(t-s)f](x,s)ds
\end{equation}
which makes the symmetry of the solution more evident.

\section{A Nonlinear Transport Equation}
Let's extend our symmetry method to attack a slightly non-linear
transport equation. Here, we will take $f$ in
\eqref{eq:ntransport-ivp} and replace it with a term proportional to
$u$ so that the rate of motion along $(b,1)$ is proportional to $u$
itself.

\begin{ndefn}(Proportional Transport IVP)

  The \textbf{proportional transport IVP} is
  \begin{equation}
    \label{eq:ptransport-ivp}
    \begin{cases}
      u_t + b\cdot D_xu + cu = 0 & \text{in } \R^n \times \R^+
      u(x,0) = g
    \end{cases}
  \end{equation}
  with $b \in \R^n,c\in \R$ are fixed and $g \in C^1$.
\end{ndefn}
The plan is to use reasoning similar to \eqref{eq:v-der} for ordinary
differential equation (ODE) inspirtation. The plan is to run the proof
of \eqref{eq:ntransport-ivp} as closely as possible.

\begin{nthm}(Solution of Proportional Transport Equation)
  The IVP \eqref{eq:ptransport-ivp} has a solution
  \begin{equation}
    \label{eq:ptransport-sol}
    u(x,t) = e^{-ct}g(x-tb)
  \end{equation}
\end{nthm}
\begin{proof}
  Like last time, lets start with a change of variables. This will
  reduce us to an ODE again. For $v(s) = u(x+sb,t+s)$ we get
  \begin{equation}
    \label{eq:trans-ode}
    \dot{v}(s) + cv(s) = 0
  \end{equation}
  We then solve the ODE giving us
  \begin{equation}
    \label{eq:ode-sol}
    v(s) = u(x+sb,t+s) = k(x,t)e^{-cs}
  \end{equation}
  then
  \begin{equation}
    \label{eq:u-sol}
    \underbrace{u(x-tb,0)}_{g(x-tb)} = k(x,t)e^{ct}
  \end{equation}
  meaning $k(x,t) = e^{-ct}g(x-tb)$. Hence,
  \begin{equation}
    \label{eq:u-sol}
    u(x+sb,t+s) = e^{-ct}g(x-tb)e^{-cs}
  \end{equation}
  Specializng to $s=0$ we arrive at the solution
  \begin{equation}
    \label{eq:ptransport-sol}
    u(x,t) = e^{-ct}g(x-tb)
  \end{equation}
\end{proof}
Again, we used ODE methods along the lines of summetry to solve the
problem. Later on, we will return to this theme of reuction to ODE for
more general, nonlinear PDE.

\section{Weak Solutions}
Another useful observation is that the nature of the proof did not
require any smoothness of the initial data, $g$ in any of the
transport equations that we have seen thus far. In what follows, we
will consider what happens when we merely require that $g$ is $C^0$ or
even merely bounded.

the idea here is to multiply the transport equation by a smooth
function $\phi(x,t)$ and integrate over the domain:
\begin{equation}
  \label{eq:i}
  I = \int_0^\infty\int_{\R^n} (u_t + b\cdot D_xu)\phi dxdt
\end{equation}
In order to ensure that $I$ is well-behaved we will additionally
requre that $\phi$ have compact support. Later on, we will call such
functions \emph{test functions}.

To make more sense of the integral \eqref{eq:i} notice that
\begin{equation}
  \label{eq:der-note}
  D(u\phi)\cdot (b,1) = (u_t + D_xu\cdot b)\phi + u(\phi_t + D_x\phi\cdot b)
\end{equation}
Now apply \eqref{eq:der-note} to \eqref{eq:i} to see
\begin{equation}
  \label{eq:i-div-form}
  I = \int_0^\infty\int_{\R^n} D(u\phi)\cdot (b,1)dxdt 
  - \int_0^\infty\int_{\R^n} u(\phi_t + D_x\phi\cdot b)dxdt
\end{equation}
Next we take advantage of the fact that $\phi$ has compact support by
hitting \eqref{eq:i-div-form} with the scalar divergence theorem
\begin{equation}
  \label{eq:scalar-div-thm}
  \int_{\Omega \subset \R^{n+1}} Dw(y)dy = \int_{\partial \Omega}w(y)\vec{n}(y)dy
\end{equation}
where $\vec{n}(y)$ is the outward unit normal field to $\partial
\Omega$. Let $B_M^+$ be a half ball with radius $M > \diam \text{supp}
\phi$.

\todo{insert diagram of what we are doing e.f. 6190 notes pg 16}

Then
\begin{equation}
  \int_0^\infty \int_{\R^n} D(u\phi)dxdt =
  \underbrace{\int_{\partial B_M^+} u\phi\cdot \vec{n}ds}_{0} +
  \int_{|x|\leq M} u\phi \vec{n}dx
\end{equation}
But $\vec{n} = (0,-1)$ so $\vec{n} \cdot (b,1) = -1$. These
considerations lead us to

\begin{ndefn}(Weak Solution of Transport Equation)
  
  We say that $u(x,t) \in C^0$ is a weak solution of the homogeneous
  transport equation if
  \begin{equation}
    \label{eq:weak-sol}
    \int_{\R^n} g(x)\phi(x,0)dx +
    \int_0^\infty\int_{\R^n} u(\phi_t +
    D_x\phi\cdot b)dxdt =
    0
  \end{equation}
  for all $C^1(\R^n \times\R^+)$ test functions $\phi$.
\end{ndefn}

\subsection{Meaning of the Weak Solution}
Ok. We have a defintion. This leads us to the question "Is this
nonsense?". At a first glance you might think that the answer is "Yes"
due to the fact that because the derivatives are passed to the test
function in \eqref{eq:weak-sol} we admit \emph{non-differentiable
  solutions}. This clearly violates the defintion we gave of classical
solutions. However, physically, these solutions seem to make
sense. Suppose that $g$ is $C^0$ and try $u(x,t) = g(x-tb)$ in
\eqref{eq:weak-sol} we get
\end{document}
