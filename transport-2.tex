\documentclass{bkcnotes}
\usepackage{bkc}

\setcourse{Partial Differential Equations}
\setnotetitle{Transport Equation II}
\setnoteimage{img/partial}

\newcommand{\todo}[1]{{\bf TODO: } #1\\}

\begin{document}
\maketitle

Last time we found the solution to the homogeneous transport IVP, This
time around, we'll extend the intuition to the nonhomogeneous problem
as well as discuss several additional points of view on the solutions
we get that will be useful in future discussions on PDEs.

\section{Nonhomogeneous Transport Equation}

\begin{ndefn}(Nonhomogeneous Transport IVP)

  The \textbf{initial value problem for the transport equation} is
  \begin{equation}
    \label{eq:ntransport-ivp}
    \begin{cases}
      (b,1) \cdot \nabla u(x,t) = f & f \in C^1, b,x \in \R^n, t \in \R^+ \\
      u = g & t = 0, g \in C^1(\R^n)
    \end{cases}
  \end{equation}
\end{ndefn}
The intuition behind this IVP is nearly identical to
\eqref{eq:transport-ivp} except now that instead of being constant
along $(b,1)$ we have an additional $f$ that we need to keep track
of. In essence we can still read off what the answer should be: The
initial value on the $\R^n$ plane as well as the total contribution of
$f$ along the line $(x,0)$ to $(x+sb,t+s)$. This leads us to

\begin{nthm}(Solution of Nonhomogeneous Transport Equation)

  The function
  \begin{equation}
    \label{eq:ntransport-sol}
    u(x,t) = g(x-tb) + \int_0^t f(x+(s-t)b,s)ds
  \end{equation}
  is the unique solution to the IVP \eqref{eq:ntransport-ivp}.
\end{nthm}
\begin{proof}
  This goes very similarly to the homogeneous case. The only
  difficulty is dealing with that $f$ term. Fortunately for us, it's
  still just calculus with a change of variables. Set
  \begin{equation}
    \label{eq:change-of-var}
    v(\tau) = u(x-\tau b,t-\tau)
  \end{equation}
\end{proof}

\end{document}
