\documentclass{bkcnotes}
\usepackage{bkc}

\setcourse{Partial Differential Equations}
\setnotetitle{Introduction}
\setnoteimage{img/partial}

\newcommand{\todo}[1]{{\bf TODO: } #1\\}

\begin{document}
\maketitle

\section{Introduction}
Partial differential equations, or PDEs for short, are are relations
between a function $u: \R^n \to \R$ and its partial
derivatives. Shortly, we will see that PDEs arise in many natural
areas of mathematical inquiry from applied mathematics and physics to
geometry and the study of complex analytic functions. In addition,
PDEs are interesting objects in their own right and present some
interesting challenges. Most of the time the question "Can we write
down a solution to this PDE?" is answered in the negative:
No. Although the outlook may look bleak, we will see that there are
many techniques available to extract as much information as possible
from a PDE and at least partially rectify our position in the struggle
to find solutions.

\section{Examples of PDEs}
We will begin by looking at some concrete examples of PDEs and where
they arise in context.

\subsection{Continuum Mechanics}
The first example that we will explore is the 1-dimensional balance
law. We imagine our domain to be a 1-dimensional interval, $I$, of
length $L$. 
\todo{Add a picture of the rod}
We are interested in the density of some given quantity
such as mass, energy, number of particles, etc. at location $x$ and
time $t \in (0,\infty)$. Let this quantity, measured in amount per
unit length be given by the function 
\[
u(x,t) : \R \times \R \to \R
\] 
To understand the change in the density we will need to quantify the
"flow" of the quantity that we are interested throughout $I$. The
\textit{flux}, $\phi(x,t)$ is the amount of quantity crossing $x$
"out" at time $t$ measure in amount per unit time. To complete our
model we imagine that the interval $I$ lives inside some ambient space
that is either adding or removing some quantity at each point of
$I$. The action of the ambient space on $I$ will be captured by some
function $f(x,t)$ called the \textit{source} (or \textit{sink})
term measure in amount per unit length per unit time.

The balance law is as follows: Fix any subsection $(a,b) \subset
I$, then
\todo{fix the spacing of the text under the braces}
\begin{equation}
  \label{eq:balance}
  \frac{d}{dt} \int_a^b u(x,t)dx =
  \underbrace{\phi(a,t)}_{\mathclap{\text{flow into } a}} -
  \underbrace{\phi(b,t)}_{\text{flow out of } b} +
  \int_a^b f(x,t)dx
\end{equation}
Assuming that $u,\phi \in C^1$ (i.e. they have continuous partial
derivatives) then ~\ref{balance} becomes
\[
\int_a^b u_t(x,t)dx = -\int_a^b \phi_x(x,t)dx + \int_a^b f(x,t)dx
\]
Collecting terms we arrive at
\[
\int_a^b u_t(x,t) + \phi_x(x,t) - f(x,t)dx = 0
\]
for all $(a,b) \subset I$. Assuming that the integrand above is
continuous then \ref{balance} holds if and only if
\begin{equation}
  \label{eq:balance-pde}
  u_t(x,t) + \phi_x(x,t) = f(x,t), x \in (0,L), t \in \R^+
\end{equation}
The PDE in \ref{balance-pde} is a local version of \ref{balance}. We
can then look at some special cases of \ref{balance-pde} to derive
some interesting balance laws.
\subsubsection{Transport Equation}
If we take $\phi = cu, c > 0$ a constant measure in length per time
then \ref{balance-pde} becomes
\[
u_t + cu_x =
\begin{cases}
  0         & \text{advection or transport equation} \\
  \lambda u & \text{advection with growth }(\lambda > 0)
              \text{ or decay }(\lambda < 0) \\
  f         & \text{forced advection}
\end{cases}
\]
Each of these assumptions lead to some form of the {\bf transport
  equation}, a classical linear PDE.

\subsubsection{Heat Equation}
Now if we let the flux be given by $\phi = - Du_x$ and $f = 0$ we
arrive at the equation
\[
u_t = Du_{xx}
\]
with $D > 0$. Physically, this choice of the flux forces movement to
go from high concentration to low. This equation is known as the {\bf}

\subsubsection{Euler's Equation}
With minimal restrictions we simply set $\phi = g(u)$ and $f = 0$ to
yield the PDE
\[
u_t + g'(u)u_x = 0
\]
When we have that $g'(u) = u$ this is {\bf Euler's equation}, a
fundamental equation from fluid mechanics.
\end{document}