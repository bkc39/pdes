\documentclass{bkcnotes}
\usepackage{bkc}

\setcourse{Partial Differential Equations}
\setnotetitle{Transport Equation I}
\setnoteimage{img/partial}

\newcommand{\todo}[1]{{\bf TODO: } #1\\}

\begin{document}
\maketitle

Ok. Now it's time to buckle down and start solving some PDEs. We will
begin with the examples that we found using conservation laws, and in
particular the transport equation. Before that, however, we need to
stop waving our hands and give some definitions.

\section{Some Definitions}

Let's define what we have been talking about

\begin{ndefn}(Partial Differential Equation)

  A \textbf{partial differential equation} (PDE) of order $k$ is a
  relation between a function and its partial derivatives of the form
  \begin{equation}
    \label{eq:pde-def}
    F(D^ku(x),D^{k-1}u(x),\ldots,Du(x),u(x),x) = 0
  \end{equation}
  Where
  \begin{equation}
    F : \R^{n^k} \times \R^{n^{k-1}} \times \cdots
    \times \R^n \times \R \times \Omega \lra \R
  \end{equation}
  is a given function.
\end{ndefn}

We also need our first notion of solution of a PDE. Later, we will
consider more relaxed notions of what it means to solve a PDE, but for
now we will go with what you would expect
\begin{ndefn}(Classical solution)

  A function $u(x)$ is a (classical) \textbf{solution} to the PDE
  \eqref{eq:pde-def} if it is a $C^n$ function in $\Omega$ with $n
  \geq k$ that satisfies \eqref{eq:pde-def} for each $x \in \Omega$.
\end{ndefn}

As you can see from the definition, everything hinges on the
properties of the given function $F$. In particular, it determines how
hard we will have to fight to find a solution, if any exist at
all. There are some special cases that we will pay particular
attention to

\begin{ndefn}(Classification of PDEs)
  \ \newline \vspace{-20px}
  \begin{enumerate}
  \item The PDE \eqref{eq:pde-def} is called \textbf{linear} if $F$ is
    linear i.e.,
    \begin{equation}
      \begin{aligned}
        F(\alpha &A^k + B^k, \alpha A^{k-1} + B^{k-1}, \ldots, \alpha A + B,x)
        = \\
        &\alpha F(A^k,A^{k-1},\ldots,A,x) + F(B^k,B^{k-1},\ldots,B,x)
      \end{aligned}
    \end{equation}
    Or equivalently,
    \begin{equation}
      \label{eq:linear}
      F(D^ku(x),D^{k-1}u(x),\ldots,Du(x),u(x),x) =
      f(x) + \sum_{|j| \leq k} \alpha_j(x)D^j(x)
    \end{equation}
    for every
    \href{http://en.wikipedia.org/wiki/Multi-index_notation}{multi-index}
    $j$.
  \item The PDE \eqref{eq:pde-def} is \textbf{semilinear} if it is
    linear in the derivatives of highest order. In symbols,
    \begin{equation}
      \label{eq:semilinear}
      f(D^{k-1}u,\ldots,Du,u,x) + \sum_{|j|}\alpha_j(x)D^ju
    \end{equation}
  \item If we allow the coefficient functions to bring nonlinearity in
    the low order derivatives i.e.,
    \begin{equation}
      \label{eq:quasilinear}
      f(D^{k-1}u,\ldots,Du,u,x) + \sum_{|j|}\alpha_j(D^{k-1}u,\ldots,Du,u,x)D^ju
    \end{equation}
    we have a \textbf{quasilinear} PDE.
  \item Finally, if we allow the nonlinearity to affect the highest
    order derivatives we have a \textbf{fully nonlinear} equation.
  \end{enumerate}
\end{ndefn}

\subsection{Examples}
The PDEs that we derived from localizing balance laws are all linear
\begin{equation}
  \label{eq:linear-examples}
  \begin{aligned}
    \lap u &= f && \text{(Poisson equation)} \\
    u_t + b \cdot Du &= 0 && \text{(Transport equation)} \\
    u_t - \lap u &= 0 && \text{(Heat Equation)} \\
    u_{tt} - \lap u &= 0 && \text{(Wave equation)}
  \end{aligned}n
\end{equation}
Some semilinear PDEs arise elsewhere in physics (and other areas, of
course)
\begin{equation}
  \label{eq:semilinear-examples}
  \begin{aligned}
    u_t + uu_x + u_{xxx} &= 0 && \text{(Korteweg-deVries, KdV, equation)} \\
    -\lap u &= f(u) && \text{(Nonlinear Poisson equation)} \\
    u_t - \lap u &= f(u) && \text{(Scalar reaction-diffusion equation)}
  \end{aligned}
\end{equation}
Quasilinear PDEs include
\begin{equation}
  \label{eq:quasilinear-examples}
  \begin{aligned}
    u_t - uu_x &= 0 && \text{(Euler ,also Burger's, equation)} \\
    u_t + \nabla \cdot F(u) &= 0 && \text{(Scalar Conservation law)}
  \end{aligned}
\end{equation}
For the true monsters which are the fully nonlinear PDEs we have
\begin{equation}
  \label{eq:nonlinear-examples}
  \begin{aligned}
    \nabla \cdot \left(\frac{\nabla u}{\sqrt{1 + |\nabla u|^2}}\right) &= 0
    && \text{(Minimal Surface equation)} \\
    u_t + H(Du,x) &= 0 && \text{(Hamilton-Jacobi equation)} \\
    u_t + u \cdot Du - \lap u &= -Dp \\
    \nabla \cdot u &= 0 && \text{(Navier-Stokes equations)}
  \end{aligned}
\end{equation}
As you might expect, the more the nonlinearity affects the higher
derivatives the more we will have to wrestle to find solutions. To
start off, we will concentrate on linear the PDEs
\eqref{eq:linear-examples} . The plan is to find some techniques that
we can extend to the more nonlinear types of equations. Each of the
techniques we will find to solve the classical linear PDEs will
eventually be extended to nonlinear PDEs which share the critical
features that make the proofs work. Let's do it to it.

\section{Transport Equation}
The goal is to solve the transport equation
\begin{equation}
  \label{eq:transport}
  \begin{aligned}
    u_t + b \cdot D_xu = 0 && \text{in } \R^n \times \R^+    
  \end{aligned}
\end{equation}
with $b \in \R^n$ fixed, for $u(x,t): \R^{n+1} \to \R$. The symbol
$D_x$ in \eqref{eq:transport} is the derivative with respect to the
spatial ($x$) variables $D_xu =
(u_{x_1},u_{x_2},\ldots,u_{x_n})$. Let's think about the geometry of
\eqref{eq:transport}.

\todo{insert diagram on characteristic line}

Written differently we see that
\begin{equation}
  \label{eq:transport-dot}
  (b,1) \cdot \nabla u = 0
\end{equation}
This means that \emph{the directional derivative in the $(b,1)$
  direction is zero.} As a result, we see that $u$ \emph{must} be
constant in the $(b,1)$ direction. Just to make sure, let's compute the
partial derivative
\begin{equation}
  \label{eq:directional-der}
  \frac{d}{ds} u(x + sb, t+s) = D_x(x+sb,t+s)\cdot b + u_t(x+sb,t+s)
\end{equation}
for fixed $(x,t)$ and all $s \in \R$. This gives us the key insight
for solving the PDE: \emph{$u$ is constant on all lines through
  $(x,t)$ parallel to the vector (b,1)}.

\subsection{Homogeneous Transport Equation}

Now we just need to prescribe the boundary values and then use the
geometry to fill in the solution for the whole space. This leads us to
the initial value problem (IVP)
\vspace{-15px}
\begin{defn}(Homogeneous Transport IVP)

  The \textbf{initial value problem for the transport equation} is
  \begin{equation}
    \label{eq:transport-ivp}
    \begin{cases}
      (b,1) \cdot \nabla u(x,t) = 0 & b,x \in \R^n, t \in \R^+ \\
      u = g & t = 0, g \in C^1(\R^n)
    \end{cases}
  \end{equation}
\end{defn}
So now we can essentially read off the solution: start on the $\R^n$
plane with value $g(x,0)$ and then follow the line $(b,1)$ retaining
the same value. In symbols

\begin{nthm}(Solution of Homogeneous Transport Equation)

  The function
  \begin{equation}
    \label{eq:transport-sol}
    u(x,t) = g(x-tb)
  \end{equation}
  is the unique solution to the IVP \eqref{eq:transport-ivp}.
\end{nthm}
\begin{proof}
  Let's just verify this is a solution. The gradient
  \begin{equation}
    \label{eq:grad-sol}
    \begin{aligned}
      \nabla g(x-tb) &= \der{}{x}g + \der{}{t}g \\
      &= Dg(x-tb) + Dg(x-tb)\cdot(-b)
    \end{aligned}
  \end{equation}
  Hence,
  \begin{equation}
    (b,1) \cdot \nabla u = b \cdot Dg(x-tb) - b\cdot Dg(x-tb) = 0
  \end{equation}
  as desired. And clearly on the $\R^n$ hyperplane we have
  \begin{equation}
    u(x,0) = g(x - 0b) = g(x)
  \end{equation}
  So $g(x-tb)$ is a valid solution.

  As for the uniqueness suppose that there were another solution
  $v(x,t)$ to the IVP. Because $v = g$ in $\R^n \times \{0\}$ we have
  that $u=v$ there as well. Because $u$ and $v$ are both constant
  along each line $(x+sb,s+t)$ they are equal on each line, and hence
  the whole space.
\end{proof}

The key observation was that \emph{the solution must be constant along
  certain curves}, in this case it was lines. Then, once given the
initial values on these special lines, we could piece together the
solution in a straighforward way. This geometric intuition will be
important in solving more complicated IVPs in the future.
\end{document}