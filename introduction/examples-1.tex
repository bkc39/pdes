\documentclass{bkcnotes}
\usepackage{bkc}

\setcourse{Partial Differential Equations}
\setnotetitle{Examples of PDEs I}
\setnoteimage{img/partial}

\newcommand{\todo}[1]{{\bf TODO: } #1\\}

\begin{document}
\maketitle

We begin with a few examples of PDEs and informally illustrate the
main techniques for finding solutions. Later on, we'll give the
rigorous justifications for these manipulations. 

\section{Heat Equation in $D \subset \R^n$}
We derived the 1-dimensional heat equation last time. Now we will see
what happens in the cases $n=2,3$. Let $D$ be an open, bounded domain
in $\R^n$ with a $C^1$ boundary. Then the balance law becomes
\begin{equation}
  \label{eq:heat-balance}
  \frac{d}{dt}\int_{D} u(x,t)dx =
  -\int_{\partial D} \phi(x,t) \cdot \vec{n}ds + \int_{D}f(x,t)dx
\end{equation}
Where $\vec{n}$ is the \textit{outward} unit normal vector. Recall the
divergence theorem
\begin{equation}
  \int_{\partial D} \phi \cdot \vec{n}ds =
  \int_{D} \nabla \cdot \phi dx
\end{equation}
Then if $e_1,\ldots,e_n$ is an orthonormal basis, we can resolve the
above into components
\begin{equation}
  \begin{aligned}
    \nabla \cdot \phi &= \nabla \cdot (\phi_1e_1 + \cdots + \phi_ne_n) \\
    &= \sum_{i=1}^n \frac{\partial \phi_i}{\partial x_i}
  \end{aligned}
\end{equation}
We then specialize the balance law as before to
\begin{equation}
  \phi = \begin{cases}
    -D\nabla u & D \in \R \\
    -A\nabla u & A \in \text{GL}_n(\R)
  \end{cases}
\end{equation}
So that the balance law reduces to
\begin{equation}
  \begin{aligned}
    \nabla \cdot (D\nabla u) &= D \nabla \cdot (u_{x_1},u_{x_2}, \ldots, u_{x_n}) \\
    &= D (u_{x_1x_1} + u_{x_2x_2} + \cdots + u_{x_nx_n}    
  \end{aligned}
\end{equation}
We then introduce the \textbf{laplacian} operator
\begin{equation}
  \label{eq:laplace-op}
  \lap = \sum_{k=1}^n u_{x_kx_k}
\end{equation}
So that we have that $\nabla \cdot \phi = D \lap u$.

\subsection{Local Version of The heat Equation}
All of the simplification happened under the integral sign. With the
same choice of $\phi$ above, if we take the equality in
\eqref{eq:heat-balance} to hold pointwise we arive at local version of
the heat equation
\begin{equation}
  \label{eq:heat}
  u_t = D\lap u + f
\end{equation}
Again, the local version of the heat equation admits a fewinteresting
special cases. The \textbf{steady-state} heat equation remove the time
derivative taking $u_t = 0$. We also recover the \textbf{Poisson
  equation}, $\lap u = f$ by choosing $\lap u = -f / D$.

\section{Wave Equation}
We can also find the wave equation by considering the total force on
our domain $D$. By Newton's law we have
\begin{equation}
  \label{eq:newton}
  \int_{\partial D} T \der{u}{\vec{n}}ds = T \int_{\partial D} \nabla \cdot \vec{n}ds
\end{equation}
Mechanics tells us that we have the equality
\begin{equation}
  \begin{aligned}
    T \int_{\partial D} \nabla \cdot \vec{n}ds &=
    \frac{d}{dt}\overbrace{\int_{D}\rho u_t(x,t)dx}^{\text{linear
        momentum}} \\
    &= \rho\int_D u_{tt}dx
  \end{aligned}  
\end{equation}
Dividing the constant $\rho$ through and applying the divergence
theorem we arrive at the balance law
\begin{equation}
  \label{eq:wave-balance}
  \int_{D} u_{tt}dx = \frac{T}{\rho}\int_{D} \nabla \cdot (\nabla u)dx
\end{equation}

\subsubsection{Local Wave Equation}
Taking the equality in \eqref{eq:wave-balance} pointwise leads us to the
\textbf{wave equation}
\begin{equation}
  \label{eq:wave}
  u_{tt} = c^2\lap u, c^2 = \frac{T}{\rho}
\end{equation}
Again, if we remove the time dependence in \eqref{eq:wave} we arrive at the
laplace equation $\lap u = 0$.

\subsection{Summary}
At this point we have derived the four classical PDEs:
\begin{enumerate}
\item Transport Equation
\item Laplace Equation
\item Heat Equation
\item Wave Equation
\end{enumerate}
as local versions of a corresponding "balance law" with physically
significant choices of the variables in our model from last
time. After looking at some more examples of PDEs in other areas of
mathematics, we will dive into the search for solutions to these PDEs.
\end{document}